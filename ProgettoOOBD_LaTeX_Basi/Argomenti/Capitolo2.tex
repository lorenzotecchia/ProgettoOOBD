\chapter{Progettazione concettuale}
    \section{Class Diagram}
        
    \section{Analisi della ristrutturazione del Class Diagram}
        
        \subsection{Analisi delle ridondanze}
            
        \subsection{Analisi degli identificativi}
            
        \subsection{Rimozione degli attributi multipli}
            
        \subsection{Rimozione degli attributi composti}
            
        \subsection{Partizione/Accorpamento delle associazioni}
            
        \subsection{Rimozione delle gerarchie}
    
    \section{Class Diagram ristrutturato}
  
    \section{Dizionario delle classi}
    Viene qui rappresentata la tabella con le classi le definizioni di tali con gli attributi delle rispettive.
\begin{table}[]
\centering
\caption{Dizionario delle Classi}
\label{tab:DizionarioClassi}
\resizebox{\textwidth}{!}{%
\begin{tabular}{|c|c|l|}
\hline
Classe   & Descrizione                                                                                                           & \multicolumn{1}{c|}{Attributi}                                                                                                                                   \\ \hline
Event    & A thing that happens or takes place                                                                                   & \begin{tabular}[c]{@{}l@{}}CodEvent\\ StartDate\\ EndDate\\ LocationPromoter\\ Manager\end{tabular}                                                              \\ \hline
Book     & A written or printed work                                                                                             & \begin{tabular}[c]{@{}l@{}}Doi\_B\\ ISBN\_B\\ Edition\\ ReleaseDate\\ PublishingHouse\\ Author\\ Language\\ AccessMode\\ Title\\ Argument\\ Reprint\end{tabular} \\ \hline
Series   & A set of books                                                                                                        & \begin{tabular}[c]{@{}l@{}}ISSN\_S\\ Curator\\ Edition\\ Name\\ AccessMode\end{tabular}                                                                          \\ \hline
Magazine & A periodical publication containing articles                                                                          & \begin{tabular}[c]{@{}l@{}}ISSN\_M\\ Name\\ Argument\\ Manager\\ YearRelease\\ PublicationPeriod\\ AccessMode\end{tabular}                                       \\ \hline
Article  & \begin{tabular}[c]{@{}c@{}}A piece of nonfictional prose that is\\  an independent part of a publication\end{tabular} & \begin{tabular}[c]{@{}l@{}}Doi\_A\\ Title\\ AccessMode\\ YearRelease\\ Editor\\ Author\end{tabular}                                                              \\ \hline
Loan     & A thing that is borrowed                                                                                              & \begin{tabular}[c]{@{}l@{}}LoanCode\\ StartLoan\\ EndLoan\end{tabular}                                                                                           \\ \hline
User     & A person who uses or operates something                                                                               & \begin{tabular}[c]{@{}l@{}}FCUser\\ Password\end{tabular}                                                                                                        \\ \hline
\end{tabular}%
}
\end{table}
        \newpage
    \section{Dizionario delle associazioni}